% mnras_template.tex
%
% LaTeX template for creating an MNRAS paper
%
% v3.0 released 14 May 2015
% (version numbers match those of mnras.cls)
%
% Copyright (C) Royal Astronomical Society 2015
% Authors:
% Keith T. Smith (Royal Astronomical Society)

% Change log
%
% v3.0 May 2015
%    Renamed to match the new package name
%    Version number matches mnras.cls
%    A few minor tweaks to wording
% v1.0 September 2013
%    Beta testing only - never publicly released
%    First version: a simple (ish) template for creating an MNRAS paper

%%%%%%%%%%%%%%%%%%%%%%%%%%%%%%%%%%%%%%%%%%%%%%%%%%
% Basic setup. Most papers should leave these options alone.
\documentclass[a4paper,fleqn,usenatbib]{mnras}

% MNRAS is set in Times font. If you don't have this installed (most LaTeX
% installations will be fine) or prefer the old Computer Modern fonts, comment
% out the following line
\usepackage{newtxtext,newtxmath}
% Depending on your LaTeX fonts installation, you might get better results with one of these:
%\usepackage{mathptmx}
%\usepackage{txfonts}

% Use vector fonts, so it zooms properly in on-screen viewing software
% Don't change these lines unless you know what you are doing
\usepackage[T1]{fontenc}
\usepackage{ae,aecompl}


%%%%% AUTHORS - PLACE YOUR OWN PACKAGES HERE %%%%%

% Only include extra packages if you really need them. Common packages are:
\usepackage{graphicx}	% Including figure files
\usepackage{amsmath}	% Advanced maths commands
\usepackage{amssymb}	% Extra maths symbols

%%%%%%%%%%%%%%%%%%%%%%%%%%%%%%%%%%%%%%%%%%%%%%%%%%

%%%%% AUTHORS - PLACE YOUR OWN COMMANDS HERE %%%%%

% Please keep new commands to a minimum, and use \newcommand not \def to avoid
% overwriting existing commands. Example:
%\newcommand{\pcm}{\,cm$^{-2}$}	% per cm-squared

%%%%%%%%%%%%%%%%%%%%%%%%%%%%%%%%%%%%%%%%%%%%%%%%%%

%%%%%%%%%%%%%%%%%%% TITLE PAGE %%%%%%%%%%%%%%%%%%%

% Title of the paper, and the short title which is used in the headers.
% Keep the title short and informative.
\title[Bulgeless galaxies, black hole spin and outflows]{Black hole spin really does affect outflows, probably}

% The list of authors, and the short list which is used in the headers.
% If you need two or more lines of authors, add an extra line using \newauthor
\author[Smethurst, Lintott and Simmons]{
Rebecca J. Smethurst,$^{1,2}$\thanks{E-mail: rebecca.smethurst@chch.ox.ac.uk (RJS)}
Chris J. Lintott,$^{2}$
and Brooke D. Simmons$^{3}$
\\
% List of institutions
$^{1}$ChristChurch, Oxford\\
$^{2}$Department of Physics, University of Oxford, Keble Road, Oxford, OX1 3RH\\
$^{3}$University of Lancaster, Up North.}

% These dates will be filled out by the publisher
\date{Accepted XXX. Received YYY; in original form ZZZ}

% Enter the current year, for the copyright statements etc.
\pubyear{2018}

% Don't change these lines
\begin{document}
\label{firstpage}
\pagerange{\pageref{firstpage}--\pageref{lastpage}}
\maketitle

% Abstract of the paper
\begin{abstract}
The relationship of the spin of supermassive blackholes and their propensity to power outflows is a much debated one. The spin of a black hole can be assumed to be a product of its accretion history; accretion from a well-ordered disk will tend to spin up the black hole less effectively that the merger of two black holes following a major merger. We use a sample of `bulgeless' galaxies which are maximally disk dominated and thus which have not undergone a major merger to test the idea that black holes with greater spins are more likely to power outflows. We show that [Haven't done the work yet]. 
\end{abstract}

% Select between one and six entries from the list of approved keywords.
% Don't make up new ones.
\begin{keywords}
keyword1 -- keyword2 -- keyword3
\end{keywords}

%%%%%%%%%%%%%%%%%%%%%%%%%%%%%%%%%%%%%%%%%%%%%%%%%%

%%%%%%%%%%%%%%%%% BODY OF PAPER %%%%%%%%%%%%%%%%%%

\section{Introduction}

Firstly, a bit about the importance of outflows.

Then we need to justify/discuss the purported connection between spin and outflows.

\citet{BensonBabul} study the relationship between black hole spin and outflows, showing that there is a balance between the spin-up due to accretion of material and the braking torque exerted by the jet. 


Then we need to say something about mergers, I would think, and how they affect black hole spin.

\cite{Gammie04} argue that following a merger between two black holes, it is reasonable to assume that the final black hole has an angular momentum that is equal to that of the binary; this is therefore a highly effective way of spinning up such a system. We should expect galaxies which have undergone major mergers, and whose central black holes are the results of mergers, to have central black holes with higher spins than otherwise would be the case. Specifically, `the merger of two black holes of comparable mass will immediately drive the spin parameter of the merged hole to 0.8' where the spin parameter is J/$\mathrm{M_{BH}}$.


Then we need to discuss bulgeless galaxies, and question whether we can say something about their black hole spin.

\section{Methods}
\subsection{Sample selection}
Including control sample
\subsection{How to spot an outflow}

\section{Results}
Including a vaguely plausible statistical test

\section{Conclusion}
This looks promising - proper work with NuStar needed

\section*{Acknowledgements}
Thanks to people. 

%%%%%%%%%%%%%%%%%%%%%%%%%%%%%%%%%%%%%%%%%%%%%%%%%%

%%%%%%%%%%%%%%%%%%%% REFERENCES %%%%%%%%%%%%%%%%%%

% The best way to enter references is to use BibTeX:

%\bibliographystyle{mnras}
%\bibliography{example} % if your bibtex file is called example.bib


% Alternatively you could enter them by hand, like this:
% This method is tedious and prone to error if you have lots of references
\begin{thebibliography}{99}
\bibitem[\protect\citeauthoryear{Benson \& Babul}{2009}]{BensonBabul}
Benson, A.J. \& Babul, A., 2009, MNRAS, 397, 1302
\bibitem[\protect\citeauthoryear{Gammie, Shapiro \& McKinney}{2004}]{Gammie04}
Others S., 2012, Journal of Interesting Stuff, 17, 198
\end{thebibliography}

%%%%%%%%%%%%%%%%%%%%%%%%%%%%%%%%%%%%%%%%%%%%%%%%%%

%%%%%%%%%%%%%%%%% APPENDICES %%%%%%%%%%%%%%%%%%%%%

\appendix

\section{Some extra material}

If you want to present additional material which would interrupt the flow of the main paper,
it can be placed in an Appendix which appears after the list of references.

%%%%%%%%%%%%%%%%%%%%%%%%%%%%%%%%%%%%%%%%%%%%%%%%%%


% Don't change these lines
\bsp	% typesetting comment
\label{lastpage}
\end{document}

% End of mnras_template.tex